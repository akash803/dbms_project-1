% ---- Don't modify from Line no. 2 to 74 ----
\documentclass[12pt]{article}

\usepackage{lineno,hyperref}
\modulolinenumbers[5]
\usepackage{graphics}
\usepackage{graphicx}
\usepackage{cite}
\usepackage{epsfig}
\usepackage{amsmath}   
\usepackage{amssymb}
\usepackage{placeins}
\usepackage[linesnumbered,ruled,vlined]{algorithm2e}
\usepackage{setspace}
\usepackage{multirow}
\usepackage[export]{adjustbox}[2011/08/13]
\usepackage{tabularx}
\usepackage{algcompatible}
\usepackage{caption}
\usepackage{epsf}
\usepackage{epstopdf}
\usepackage{subfigure} 
\usepackage{colortbl}
\usepackage{longtable}
\usepackage{enumerate}
\usepackage{tabularx, booktabs}

\usepackage[table,xcdraw]{xcolor}

\usepackage{tikz}
\usepackage{multirow}
\usepackage{enumitem}
\usepackage{soul}
\usepackage{xcolor}
\usepackage[utf8]{inputenc}
\usepackage{placeins}
\usepackage{makecell}
\newcounter{qcounter}
\usepackage{tcolorbox}
\usepackage{lscape}
\usepackage{url}
\usepackage{hyperref}
\usepackage{tablefootnote}
\usepackage{url}
\usepackage{geometry}
 \geometry{
 a4paper,
 total={170mm,257mm},
 left=20mm,
 top=20mm,
 }

\usepackage{hyperref}
\hypersetup{
    colorlinks=true,
    linkcolor=blue,
    filecolor=magenta,      
    urlcolor=cyan,
}

\setlength{\parindent}{4em}
\setlength{\parskip}{1em}
\renewcommand{\baselinestretch}{1.5}

\usepackage[numbers]{natbib}
\bibliographystyle{unsrtnat}

\begin{document}

% ------------ Don't modify anything up to here ---------------

% From here on-wards modify only the relevant fields, such as Title (line no. 76), "Section:", "Course Instructor:", and "Team Members:" field. (Team Members details should be in the format such as, name, reg. no., mobile no. and email id.). Further, "Title:" can be changed as per your selected topic name. In the brief description field you can describe your topic in 250 words. Additionally, in the "Key Feature:" field add your mini-project feature as an item (example is shown). AT the end in the "Reference:" field add the website/paper/article referred for this mini-project as an item.

\begin{center}
    \textbf{\Large{Abstract \\
    (\textcolor{black}{Payroll Management System})}}
\end{center}

\noindent 
\textbf{Course Code:} CS254
\hspace{1.97in} 
\textbf{Course Title:} DBMS LAB \\
\textbf{Semester:} B. Tech 4$^{th}$ Sem 
\hspace{1.59in} 
\textbf{Section:} S1 and S2 \\
\textbf{Academic Year:} 2020-21 
\hspace{1.75in} 
\textbf{Course Instructor:} Dr. Annappa B and Mr. Sharath Yaji \\
\textbf{Team Members:} \\
\textbf{1.} Sanjkeet Jena, 191CS246, 7735778182, iamkeet.191cs246@nitk.edu.in 
\newline
\textbf{2.} Akash Meena, 191CS206, 8824257029, meenaakash1110@gmail.com
\newline
\textbf{3.} Pranav DV, 191CS234, 7760785980, pranavdv.191cs234@nitk.edu.in

\vspace{0.25in}

\noindent
\textbf{Brief Description:}
\newline
\newline
Payroll Management System is an important area of any business. It allows for a company to manage employee's expenses, salary, gross salary, deduction, tax and other things for a given time period. The two essential components of this system are management and accounting. Payroll is a major concern for any company since it is mandatory to pay every employee according to the governments rules and regulations otherwise there would be serious financial and legal consequences. This system facilitates a company to handle all the legal processes and an employee's expenditure in a systematic manner.
\newline
\newline
This payroll system is designed to all tasks of employee payment and filing of the required taxes. These tasks include keeping track of leaves, calculating wages, taxes, deduction and bonuses, as well as printing and delivering cheques to employees. This system does not require too much input from the user (the administrator) since once the details like base salary, department, grade etc. have been entered, the system automatically calculates all other information.
\newline
\newline
The purpose of the system is to implement the following:
\begin{itemize}
    \item Manage employee information efficiently
    \item Define the salary, deductions, leaves etc.
    \item Conveniently generate payslips
    \item Generate and manage salary structure of employees
    \item Generate reports related to employees
\end{itemize}
\noindent
The only users authorized to access this system are administrators and this is done using a valid username and password. Admins can add new employees, departments and pay grades. They also control the duration of time for which employees have been employed under specific pay grades. Apart from this, they can also view past data of any employee and generate their monthly salaries via an automated process.
\newline
\newline
\noindent
Some of the advantages of this system are:
\begin{enumerate}
    \item It saves time and speeds up virtually every aspect through a range of automated services
    \item It is secure as only admins are authorized to access and modify the information
    \item It is user friendly, the admins need not know the technicalities of the payroll is calculated and no former knowledge is required due to its intuitive nature
    \item There is very little scope for any errors in processing since it is fully automated
\end{enumerate}
\noindent
Overall, this software has been designed specifically to cater to a company's employee management, is self contained and words efficiently. It uses a simple database so the requirements are feasible and provides an intuitive graphical user interface for easy accessibility. Finally, automating this process save the admin or manager a lot of time and resources. 

\noindent
\textbf{Key Features:}
\begin{enumerate}
    \item Login facility
    \item Add new employee
    \item Add new department
    \item Add new pay grades
    \item Manage employee grade and department
    \item Manage employee salary details
    \item Generate payroll
\end{enumerate}
\textbf{Software Specifications:}
\begin{itemize}
\item Frontend: HTML, CSS, JavaScript(React Framework) 
\item Backend: Node.js, MySQL
\end{itemize}

\noindent
\textbf{References:}
\begin{enumerate}
    \item\href{https://github.com/fenil29/employee-management-system-frontend-react}{Sample code base for frontend}
    \item\href{https://github.com/fenil29/employee-management-system-backend-node}{Sample Code base for backend}
    \item\href{https://github.com/silentarrowz/payroll-react}{Sample code base using create react app}
    \item\href{https://medium.com/valtech-ch/setup-a-rest-api-with-sequelize-and-express-js-fae06d08c0a7}{REST Api to connect frontend to backend}
    \item\href{https://reactjs.org/tutorial/tutorial.html}{React API tutorial}
    \item\href{https://www.mysqltutorial.org/mysql-nodejs/connect/}{Connecting MySQL using Node.js}
    \item\href{https://www.w3schools.com/nodejs/}{Node.js tutorial}
    
\end{enumerate}

\begin{center}
    \textbf{**** END ****}
\end{center}

\end{document}
